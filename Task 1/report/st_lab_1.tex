% XeLaTeX document
\documentclass[12pt, a4paper]{article}
% Fonts, codecs
\usepackage{fontspec}
\usepackage{polyglossia}

\usepackage{hyperref}

% Code listings
\usepackage[cache=false]{minted}
\usepackage{listings}
\usepackage{lstfiracode}

\usepackage{cite}

\usepackage{amsmath, amsfonts, amssymb, amsthm, mathtools}
\usepackage{titletoc}
\usepackage{graphicx}
\usepackage{subfigure}

\usepackage{titlesec}
\usepackage{booktabs}
\usepackage{caption} 

\setdefaultlanguage{russian}
\setotherlanguage{english}

% Main document font
\setmainfont{CMU Serif}
\newfontfamily\cyrillicfont{CMU Serif}[Script=Cyrillic]

% Code highlighting and listings
\usemintedstyle{colorful}
\newenvironment{code}{\captionsetup{type=listing}}{}

% Font for listings
\setmonofont{[FiraCode-Regular.otf]}[
  Contextuals=Alternate  % Activate the calt feature
]

\graphicspath{{Figures/}}
\DeclareGraphicsExtensions{.pdf,.png,.jpg}





\usepackage{fontspec}
\setlength{\textheight}{8.5in}
\setlength{\textwidth}{6.6in}
\setlength{\headheight}{0in}
\setlength{\headsep}{0in}
\setlength{\topmargin}{0in}
\setlength{\oddsidemargin}{0in}
\setlength{\evensidemargin}{0in}
\setlength{\parindent}{.3in}

\renewcommand{\baselinestretch}{1.4} 

\begin{document}

\begin{titlepage}

\begin{center}
\large{Санкт-Петербургский политехнический университет Петра Великого\\
Институт прикладной математики и механики\\
Высшая школа прикладной математики и вычислительной физики\\}
\end{center}


\vspace{4cm}

\begin{center}
{\huge {\textbf{Многомерный статистический анализ}}}\\

\bigskip 
\large{{Отчет по лабораторной работе} \\
Тема: {Регрессионный анализ}}
\end{center}

\vspace{2.5cm}

\begin{flushleft}
    \begin{minipage}{0.35\textwidth}
        \textbf{Выполнил:} \\
        студент гр. 3630102/60401 \\ {Камалетдинова Ю. А.} \\ 

        \vspace{0.5cm} 
        \textbf{Проверил:} \\
        к. ф-м. н., доцент \\ {Павлова Л. В.}
        \vspace{1cm}
    \end{minipage}

\end{flushleft}

\vspace{0.5cm}

\begin{center}
Санкт-Петербург\\
2020
\end{center}

\end{titlepage}


\tableofcontents
\addtocontents{toc}{~\hfill\par}
\vfill ~
\setcounter{section}{0}


%%%%%%%%%%%%%%%%%%%%%%%%%%%%%%%%%%%%%%%%%%

\newpage 
\section*{Постановка задачи}
\addcontentsline{toc}{section}{Постановка задачи}

\indent{\indent Цель лабораторной работы –– проанализировать выборку, построить и обосновать модель закона распределения, которой подчиняются данные.}

%%%%%%%%%%%%%%%%%%%%%%%%%%%%%%%%%%%%%%%%%%

\section{Ход работы}

\indent{\indent Первоначально, до выдвигания гипотезы о распределении, необходимо визуализировать данные. Они представляют собой вещественнозначную выборку $\{x\}_{i = 1}^n$ объема $n = 60$, все значения положительны. Границы выборки равны $x_{min} \sim 0.005$ и $x_{max} \sim 5.032 $. Гистограмма и эмпирическая функция распределения представлены на рисунке \ref{fig:1_hist_edf}.}

\begin{figure}[h!]
	\centering
	\includegraphics[width=12cm, height=7cm]{1_hist_edf}
	\caption{Гистограмма и эмпирическая функция распределения набора данных}
	\label{fig:1_hist_edf}
\end{figure}

\indent{ Гистограмма по выборке \ref{fig:1_hist_edf}, очевидно, описывается асимметричным распределением. Вычислим некоторые статистки по формулам \ref{eq:mean} - \ref{eq:kurt}. По результатам из таблицы \ref{tab:stats} можно сказать, что данные сильно положительно скошены ($\gamma_1$ > 1). Значение коэффициента эксцесса указывает на то, что перед нами распределение с тяжелыми хвостами.}


\begin{table}[htb]   
	\centering 
	\begin{tabular*}{0.5\textwidth}{@{\extracolsep{\fill} } cccc}
		\toprule
		$\overline{x}$ & $s$ & $\gamma_1$ & $\gamma_2$ \\ 
		\midrule
		0.948 & 1.126 & 1.631 & 5.500 \\ 
		\bottomrule
	\end{tabular*}
	\caption{Статистики по набору данных}
	\label{tab:stats}
\end{table}

\begin{equation}
	\overline{x} = \frac{1}{n}\sum_{i=1}^{n} x_i
	\label{eq:mean}
\end{equation}

\begin{equation}
	s = \sqrt{\frac{1}{n - 1}\sum_{i=1}^{n} (x_i - \overline{x}}
	\label{eq:s}
\end{equation}

\begin{equation}
	\gamma_1 = \frac{\mu_3}{\mu_2^{3/2}}
	\label{eq:skew}
\end{equation}

\begin{equation}
	\gamma_2 = \frac{\mu_4}{\mu_2^2} - 3
	\label{eq:kurt}
\end{equation}

\begin{equation}
	f(x; k) = \frac{1}{2^{(k/2)} \Gamma(k / 2)} x^{k/2 - 1}e^{-x / 2}, \; k = 1, 2, \dots
	\label{eq:chi}
\end{equation}

\vspace{1cm}

\indent{ Критерий хи-квадрат требует, чтобы интервалы разбиения содержали не менее чем 5 наблюдений \ref{fig:hist_4_bins}.}

\begin{figure}[h!]
	\centering
	\includegraphics[width=9cm, height=5cm]{hist_4_bins}
	\caption{Гистограммы с интервалами, где частоты попадания наблюдений $\ge 5$}
	\label{fig:hist_4_bins}
\end{figure}

\vspace{1cm}

\indent{ Для сравнения распределений применим критерий $\chi^2$. Сложная гипотеза $H_0$ звучит так: наблюдения $\{x\}_{i = 1}^n$ порождаются функцией $F(x, \theta), \; \theta \in \mathcal{R}^d$. Для этого посчитаем вероятности попадания в интервалы разбиения, зная истинные функции предполагаемых распределений, и минимизируем статистику $\chi^2$ \ref{eq:chi_stat} по параметру $\theta$. Поскольку гистограмма по построенным данных явно несимметрична, будем проверять гипотезу для несимметричных распределений, а именно: логнормальное, экспоненциальное, гамма-распределение.}

\indent{ Критическое значение статистики $\chi^2$ с 2 степенями свободы для правостороннего теста на уровне значимости 0.05 –– 5.991, с 1 степенью свободы –– 3.841. Приведем полученные значения статистки с оптимизированными параметрами распределения в таблице \ref{tab:p_val}.}

\begin{equation}
	\chi^2 = \sum_{j=1}^{k}\frac{(n_j - E_j)^2}{E_j} \sim \chi_{k-d-1}^2, \; k = 4, \; d = 1, 2
	\label{eq:chi_stat}
\end{equation}

\begin{table}[htb]   
	\centering 
	\begin{tabular*}{0.85\textwidth}{@{\extracolsep{\fill} } lcl}
		\toprule
		Вид распределения & Статистика $\chi^2$ & Параметры  \\ 
		\midrule
		$Lognorm$ & 1.667 $\le 3.841$  & logmean = -0.41, logsd = 1.439 \\ 
		$Exponential$ & 3.842 $\le 5.991$ & rate = 0.942 \\ 
		$Gamma$ & 1.066 $\le 3.841$ & shape = 0.564, scale = 1.993\\
		\bottomrule
	\end{tabular*}
	\caption{Оптимизированные статистики для  распределений}
	\label{tab:p_val}
\end{table}

\indent{ Таким образом, все статистики в таблице \ref{tab:p_val} меньше соответствующих критических значений критерия, и гипотезы не могут быть отклонены. Наименьшее значение статистики было получено в случае гамма-распределения. Посмтроим график функции плотности распределения с найденными параметрами и гистограмму данных \ref{fig:gamma_hist}.}

\begin{figure}[h!]
	\centering
	\includegraphics[width=14cm, height=8cm]{gamma_hist}
	\caption{Гистограммы данных и плотность гамма-распределения}
	\label{fig:gamma_hist}
\end{figure}

%%%%%%%%%%%%%%%%%%%%%%%%%%%%%%%%%%%%%%%%%%

\section{Реализация}

\indent{\indent Для расчетов использовались библиотеки языка программирования \textbf{R}.}
%%%%%%%%%%%%%%%%%%%%%%%%%%%%%%%%%%%%%%%%%%

\section*{Заключение}
\addcontentsline{toc}{section}{Заключение}

\indent{\indent В ходе анализа выборки были выдвинуты и проверены сложные гипотезы о нескольких распределениях. Наиболее блинзким оказалось гамма-распределение с параметрами $k = 0.564, \theta = 1.993$.}

\clearpage

\bibliographystyle{splncs04}
\bibliography{mybibliography}


\end{document}{}