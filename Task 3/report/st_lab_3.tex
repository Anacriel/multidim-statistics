% XeLaTeX document
\documentclass[12pt, a4paper]{article}
% Fonts, codecs
\usepackage{fontspec}
\usepackage{polyglossia}

\usepackage{hyperref}

% Code listings
\usepackage[cache=false]{minted}
\usepackage{listings}
\usepackage{lstfiracode}

\usepackage{cite}

\usepackage{amsmath, amsfonts, amssymb, amsthm, mathtools}
\usepackage{titletoc}
\usepackage{graphicx}
\usepackage{subfigure}

\usepackage{titlesec}
\usepackage{booktabs}
\usepackage{caption} 

\setdefaultlanguage{russian}
\setotherlanguage{english}

% Main document font
\setmainfont{CMU Serif}
\newfontfamily\cyrillicfont{CMU Serif}[Script=Cyrillic]

% Code highlighting and listings
\usemintedstyle{colorful}
\newenvironment{code}{\captionsetup{type=listing}}{}

% Font for listings
\setmonofont{[FiraCode-Regular.otf]}[
  Contextuals=Alternate  % Activate the calt feature
]

\graphicspath{{Figures/}}
\DeclareGraphicsExtensions{.pdf,.png,.jpg}





\usepackage{fontspec}
\usepackage{diagbox}
\setlength{\textheight}{8.5in}
\setlength{\textwidth}{6.6in}
\setlength{\headheight}{0in}
\setlength{\headsep}{0in}
\setlength{\topmargin}{0in}
\setlength{\oddsidemargin}{0in}
\setlength{\evensidemargin}{0in}
\setlength{\parindent}{.3in}

\renewcommand{\baselinestretch}{1.4} 

\begin{document}

\begin{titlepage}

\begin{center}
\large{Санкт-Петербургский политехнический университет Петра Великого\\
Институт прикладной математики и механики\\
Высшая школа прикладной математики и вычислительной физики\\}
\end{center}


\vspace{4cm}

\begin{center}
{\huge {\textbf{Многомерный статистический анализ}}}\\

\bigskip 
\large{{Отчет по лабораторной работе} \\
Тема: {Регрессионный анализ}}
\end{center}

\vspace{2.5cm}

\begin{flushleft}
    \begin{minipage}{0.35\textwidth}
        \textbf{Выполнил:} \\
        студент гр. 3630102/60401 \\ {Камалетдинова Ю. А.} \\ 

        \vspace{0.5cm} 
        \textbf{Проверил:} \\
        к. ф-м. н., доцент \\ {Павлова Л. В.}
        \vspace{1cm}
    \end{minipage}

\end{flushleft}

\vspace{0.5cm}

\begin{center}
Санкт-Петербург\\
2020
\end{center}

\end{titlepage}


\tableofcontents
\addtocontents{toc}{~\hfill\par}
\vfill ~
\setcounter{section}{0}


%%%%%%%%%%%%%%%%%%%%%%%%%%%%%%%%%%%%%%%%%%

\newpage 
\section*{Постановка задачи}
\addcontentsline{toc}{section}{Постановка задачи}

\indent{\indent В данной работе рассматривается задача классификации объектов на основе дискриминантного анализа. . Пусть есть некоторые популяции $W_1, W_2$. Тогда задача состоит в отнесении объекта $w$ на основе вектора признаков $X = {x_1, \dots, x_p}$. Необходимо построить и протестировать классификатор на разных данных:

\begin{enumerate}
	\item Модельные данные
	\item Данные из репозитория \\ \url{https://archive.ics.uci.edu/ml/machine-learning-databases/statlog/german/}
\end{enumerate}}

%%%%%%%%%%%%%%%%%%%%%%%%%%%%%%%%%%%%%%%%%%

\section{Ход работы}
\subsection{Построение дискриминантной функции}

\indent{\indent Сгенерируем обучающие выборки, представляющие популяции $W_1, W_2$, в предположении, что выборки имеют многомерное нормальное распределение:}

\[x \sim N\left(\mu_1, \sum\nolimits_1\right), \;\; x \sim N\left(\mu_2, \sum\nolimits_2\right)\],
где $\mu_1, \mu_2 \in R^p$ –– векторы средних, $\sum\nolimits_1 = \sum\nolimits_2 \in R^p \times R^p$ –– матрицы ковариаций.

\vspace{1cm}

\indent{Дискриминантная функция имеет вид}

\begin{equation}
	z(x) = \alpha_1 x_1 + \dots + \alpha_p x_p,
	\label{eq:discrim_func}
\end{equation}
где вектор коэффициентов $\alpha \in R^p$ находится по формуле:

\begin{equation}
	\alpha = \sum\nolimits^{-1} (\mu_1 - \mu_2)
	\label{eq:alpha}
\end{equation}

\indent{Наблюдение из $x_i$ из выборки будем относить к $W_1$, если $z \ge c$, и к $W_2$, если $z < c$, где $c$ –– постоянная.}

\indent{Константу $c$ будем вычислять по формулам:}

\begin{equation}
	c = \frac{(\xi_1 - \xi_2)^2}{\sigma_z^2},
	\label{eq:c_formula}
\end{equation}
где $\xi_1 = \sum_{j=1}^p \alpha_j \mu_{1_j}, \;\; \xi_2 = \sum_{j=1}^p \alpha_j \mu_{2_j}$ –– средние $z$ для $x \in W_1, W_2$ соответственно, $\sigma_z^2 = \sum_{m = 1}^p \sum_{j = 1}^p \alpha_m \sigma_{mj} \alpha_j$ –– дисперсия

\vspace{1cm}

\indent{Для измерения "расстояния" между двумя популяциями воспользуемся расстоянием Махалонобиса:}

\begin{equation}
	\Delta^2 = \frac{(\xi_1 - \xi_2)^2}{\sigma_z^2}
	\label{eq:mahalonobis}
\end{equation}

\indent{Итак, для решения задачи требуется найти коэффициенты $\alpha_1, \dots, \alpha_p$, минимизарующие $\Delta^2$. Для классификации вектора $X$ будем использовать следующее правило:}

\begin{equation}
	\begin{aligned}
	x_i \in W_1: \sum_{i = 1}^p \alpha_i x_i \ge \frac{\xi_1 + \xi_2}{2} + \ln{\frac{q_2}{q_1}} \\
	x_i \in W_2: \sum_{i = 1}^p \alpha_i x_i < \frac{\xi_1 + \xi_2}{2} + \ln{\frac{q_2}{q_1}},
	\end{aligned}
	\label{eq:decision_rule}
\end{equation}
где $P(x_i \in W_1) = q_1, \;\; P(x_i \in W_2) = q_2$ –– априорные вероятности

\subsection{Численный эксперимент}

\indent{\indent Проведем два эксперимента на смоделированных данных: в случае, когда данные "хорошо" разделимы и наоборот. Будем считать, что данные "хорошо" разделимы, когда их диапазоны значений векторов средних $\mu$ не пересекаются с учетом дисперсий.} 

\indent{Смоделируем две обучающие выборки объемами $n_1, n_2$ из нормального трехмерного распределения с заданными параметрами: $X_1 \sim N(\mu_1, \sum\nolimits_1), \; X_2 \sim N(\mu_2, \sum\nolimits_2), \; X_1, X_2 \in R^3$. Сгенерируем тестовую выборку, распределенную по трехмерному нормальному распределению, включающую наблюдения из обеих популяций. Объемы выборок $n_1 = n_2 = 150$. Приведем значения $\mu_1, \; \mu_2, \; \sum\nolimits$ для "хорошо" разделимых данных}

\begin{table}[htb]   
	\begin{minipage}{.4\linewidth}
		\begin{tabular}{| c | c | c |}
			\hline
			1.862 & 0.943 & 0.007 \\ \hline
			0.943 & 0.521 & -0.061 \\ \hline
			0.007 &	-0.0614 & 0.108 \\
			\hline
		\end{tabular}
		\caption{Матрица ковариаций для "хорошо" разделимых данных}
		\label{tab:well_sigma}
	\end{minipage}
	\hfill
	\begin{minipage}{.4\linewidth}
		\begin{tabular}{| c || c | c | c |}
			\hline
			$\mu_1$ & 3.727 & 3.324 & 3.204 \\ \hline
			$\mu_2$ & 1.365 & 1.235 & 1.030 \\
			\hline
		\end{tabular}
		\caption{Векторы средних для "хорошо" разделимых данных}
		\label{tab:well_mu}
	\end{minipage}
\end{table}

\indent{Сформируем тестовую выборку из смеси данных, распределенных как исходные популяции $W_1, W_2$. Зададим априорные вероятности $q_1 = 0.4, \; q_2 = 0.6$. Объем тестовой выборки $n_{test} = 100$. Представим результат классификации в виде матрицы ошибок:}

\begin{table}[htb]  
    \centering
	\begin{tabular}{|l||*{2}{c|}} \hline
		\backslashbox{Real}{Pred}
		&\makebox[3em]{$W_1$}&\makebox[3em]{$W_2$}\\\hline\hline
		$W_1$ & 40 & 0 \\\hline
		$W_2$ & 0 & 60 \\\hline
	\end{tabular}
	\caption{Матрица ошибок для "хорошо" разделимых данных}
	\label{tab:well_confusion}
\end{table}

\indent{Исходя из заданных $q_1, \;, q_2, \; n_{test}$ 	получаем, что в тестовой выборке 40 наблюдений популяции $W_1$ и 60 наблюдений популяции $W_2$. По результатам таблицы \ref{tab:well_confusion} делаем вывод, что все объекты были безошибочно классифицированы.}

\indent{Сгенерируем "плохо" разделимые данные: значения векторов средних $\mu$ пересекаются. Тестовую выборку будем формировать как раньше: $q_1 = 0.4, \; q_2 = 0.6$, $n_{test} = 100, \; n_1 = n_2 = 150$. Приведем параметры для генерации данных и матрицу ошибок в виде таблиц:}

\begin{table}[htb]   
	\begin{minipage}{.4\linewidth}
		\begin{tabular}{| c | c | c |}
			\hline
			1.658 & -0.333 & 1.179 \\ \hline
			-0.333 & 1.683 & -0.756 \\ \hline
			1.179 &	-0.756 & 1.101 \\
			\hline
		\end{tabular}
		\caption{Матрица ковариаций для "плохо" разделимых данных}
		\label{tab:badly_sigma}
	\end{minipage}
	\hfill
	\begin{minipage}{.4\linewidth}
		\begin{tabular}{| c || c | c | c |}
			\hline
			$\mu_1$ & 1.366 & 1.514 & 1.272 \\ \hline
			$\mu_2$ & 1.268 & 1.126 & 1.083 \\
			\hline
		\end{tabular}
		\caption{Векторы средних для "плохо" разделимых данных}
		\label{tab:badly_mu}
	\end{minipage}
\end{table}

\begin{table}[htb]  
    \centering
	\begin{tabular}{|l||*{2}{c|}} \hline
		\backslashbox{Real}{Pred}
		&\makebox[3em]{$W_1$}&\makebox[3em]{$W_2$}\\\hline\hline
		$W_1$ & 20 & 20 \\\hline
		$W_2$ & 10 & 50 \\\hline
	\end{tabular}
	\caption{Матрица ошибок для "плохо" разделимых данных}
	\label{tab:badly_confusion}
\end{table}

\indent{По результатам из таблицы \ref{tab:well_confusion} виидим, что из 100 наблюдений тестовой выборки только 70 удалось классифицировать верно.}
\clearpage

\begin{figure}[h!]
	\centering
	\includegraphics[width=8cm, height=8cm]{y_y_hat}
	\caption{Оцененные и реальные значения редуцированной ($r$) модели}
	\label{fig:y_y_hat_reduced}
\end{figure}

\section*{Заключение}
\addcontentsline{toc}{section}{Заключение}


\clearpage

\bibliographystyle{splncs04}
\bibliography{mybibliography}


\end{document}{}